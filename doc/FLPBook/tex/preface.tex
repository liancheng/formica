%!TEX root = main.tex
% revision is done
\Chapterr{Предисловие}
\small

Это пособие создавалось, как помощь студентам для проведения лабораторных работ по курсу «Функциональное и логическое программирование». Оно содержит минимум теоретического материала и, в основном, освещает практические вопросы, возникающие при изучении курса. 

Учебники по функциональному программированию условно можно разделить на две группы: одни используют для примеров языки семейства \Lang{ML} (\Lang{OCaml}, \Lang{Hope}, \Lang{Haskell} и т.д.), другие основываются на семействе \Lang{Lisp} (чаще всего, \Lang{Common Lisp} или \Lang{Scheme}), несмотря на то, что \Lang{Lisp} не является чисто функциональным. Мы пошли по второму пути: для всех иллюстраций и примеров используется язык \Scheme~--- диалект языка \Racket, разработанный специально для этого курса. В свою очередь \Racket является потомком замечательного языка \Lang{Scheme}.

Язык \Lang{Racket} был выбран по следующим причинам. Минимальность основных средств этого языка и его гибкость позволяют сосредоточиться на алгоритмах и принципах функционального программирования, практически не отвлекаясь на изучение самого языка. Расширяемость \Racket позволяет с помощью одних только базовых средств построить многие специфические для функционального программирования конструкции, такие как ленивые потоки, монады, абстрактные алгебраические типы, продолжения и т.\,п. Рассмотрение и построение этих концепций <<с нуля>> позволяет глубже понять их принципы и грамотно применять их в <<серьёзных>> языках, таких как \Lang{Haskell} или \Lang{Erlang}.
Сам \Lang{Racket} является полноценным, активно развивающимся языком программиррования общего назначения, предназначенным для быстрого прототипирования, разработки новых экспериментальных концепций теоретического программирования, создания веб-серверов и приложений и т.д. Он обладает очень богатым инструментарием и замечательно документирован. Кроме того, для этого языка существует удобная свободно распространяемая интегрированная среда разработки \Lang{DrRacket}. Не последнюю роль сыграло и то, что именно \Lang{Scheme} используется в замечательных фундаментальных учебниках по общему программированию: <<Структура и интерпретация компьютерных программ>>, Х. Абельсон и др. или <<Как разрабатывать программы>>, М. Феллайзен и др. 

Мы намеренно используем только очень незначительную часть инструментария, предоставляемого языком \Racket. К примеру, мы не будем рассматривать разнообразные специфические типы данных, которые представлены в языке: структуры, векторы, объекты и пр., не будем пользоваться ключами при определении и использовании функций, ограничимся написанием только самых примитивных макросов. Кроме базовых функций и форм языка, мы воспользуемся подстановками с механизмом сопоставления с образцом и бесточечной нотацией, которые позволят нам писать более прозрачный код и помогут читателю освоить такие языки, как \Lang{OCaml}, \Lang{Haskell}, \Lang{Prolog}, \Lang{Mathematica}~и~т.\,п.
 
Мы не собираемся учить языку \Racket или \Lang{Scheme}; наша задача~--- дать практические навыки функционального программирования, по существу, в отрыве от какого-либо конкретного языка. И как раз для этой задачи \Racket подходит замечательно. Все главы, в которых вводятся новые понятия и концепции функционального программирования, имеют раздел, в котором указывается, в каких ещё распространённых языках программирования можно ими пользоваться.

Книга предназначена, в первую очередь, для студентов информационно\-/технических специальностей, как учебное пособие. Но она будет полезна для преподавателей, в качестве дидактического материала, а так же для всех, интересующихся программированием и желающих получить представление о функциональном программировании.